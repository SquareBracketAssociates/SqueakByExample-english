% $Author$
% $Date$
% $Revision$
%=================================================================
\ifx\wholebook\relax\else
% --------------------------------------------
% Lulu:
	\documentclass[a4paper,10pt,twoside]{book}
	\usepackage[
		papersize={6in,9in},
		hmargin={.75in,.75in},
		vmargin={.75in,1in},
		ignoreheadfoot
	]{geometry}
	\input{../common.tex}
	\pagestyle{headings}
	\setboolean{lulu}{true}
% --------------------------------------------
% A4:
%	\documentclass[a4paper,11pt,twoside]{book}
%	\input{../common.tex}
%	\usepackage{a4wide}
% --------------------------------------------
    \graphicspath{{figures/} {../figures/}}
	\begin{document}
	\renewcommand{\nnbb}[2]{} % Disable editorial comments
	\sloppy
\fi
%=================================================================
\chapter{Introspecting and metaprogramming}\label{cha:metaprog}

Like all Smalltalk dialects, \sq is a reflective programming language.
For a programming language, being reflective means that it allows both \emph{introspection} and \emph{intercession}.
Introspection is the ability to \emph{look} at the data structures that define the language itself, like objects, classes, the execution stack.
Intercession is the ability to \emph{modify} these structures, in other words to change the language semantics and the behavior of a program from within the program itself.
For instance, this allows to add spy methods, or to manipulate the execution stack from a debugger.
Note that while the Java language defines a ``reflective'' interface, it's not fully reflective, because it only provides a limited introspection.

In this chapter, you will read about uses of introspection that are useful for day-to-day programming.


\section{Under the hood: instances} % (fold)

Using the inspector, you can look at an object, change the values of its instance variables, and even send messages to it. Execute the following code in a workspace:
\begin{script}[inspectobject]{Inspecting an object}
| w |
w := Workspace new.
w openLabel: 'myworkspace'.
w inspect
\end{script}

This will open another workspace and an inspector.
The inspector shows the internal state of this new workspace, listing its instance variables in the left part (\ct!dependents!, \ct!contents!, \ct!bindings!...) and the value of the selected instance variable in the right part.
The \ct!contents! instance variable represents whatever the workspace is displaying in its text area, so i you select it, the right part will show an empty string.
Now type \ct!'1+2'! in place of that empty string, then \emph{accept} it.
The value of the \ct!contents! variable changed, but the workspace window didn't notice it, so it didn't redisplay itself.
To trigger the window refresh, evaluate \ct!self contentsChanged! in the lower part of the inspector. \damien{the french version says that the refresh didn't happen because \ct!contents:! doesn't trigger it, but the inspector bypasses the accessor...}

How does the inspector work?
In Smalltalk, all instance variables are protected.
In theory, it is thus impossible to access them if the class doesn't define any accessor.
In practice, the inspector does access instance variables without needing accessors, because it uses the reflective abilities of \st.
In \st, classes define instance variables either by name or by numeric indexes.
The inspector uses methods defined by the \ct!Object! class to access them: \ct!instVarAt: index! and \ct!instVarNamed: aString! to get the value of the instance variable at position \ct!index! or identified by \ct!aString!, respectively; to assign new values to these instance variables, it uses \ct!instVarAt:put:! and \ct!instVarNamed:put:!.
For instance, you can change the value of the \ct!w! instance variable of the workspace with: \ct!w instVarNamed: 'contents' put: '3+4'; contentsChanged!.

Please note that these methods are useful to build development or debugging tools, but using them to develop usual applications is a bad idea: these reflective methods break the encapsulation principles of object-oriented programming.

Both methods \ct!instVarAt:! and \ct!instVarAt:put:! are primitive methods, meaning that they invoke a primitive operation of the \sq virtual machine.
If you consult the code of these methods, you will see this invocation uses the strange syntax \ct!<primitive: xx>! where \ct!xx! is an integer.
There is a limited number of primitive operations ---primitives for short--- and they constitute the basic actions in \sq: basic operations on integers, accessing input/output peripherals (disk, screen, network...).
Some of these primitives also accelerate system functions.
It is also possible to define custom primitives and to link them to plugins written in C to benefit from external libraries or from the speed of a native implementation.

Here is the code of the method \ct!instVarAt:! defined by \ct!Object!:
\begin{method}[instVarAt]{Exemple of primitive method}
Object>>>instVarAt: index
  "Primitive. Answer a fixed variable in an object. The numbering of variables corresponds to the named instance variables. Fail if the index is not an Integer or is not the index of a fixed variable. Essential. See Object documentation whatIsAPrimitive."
  <primitive: 73>
  "Access beyond fixed variables."
  ^self basicAt: index - self class instSize
\end{method}

Typically, the code after the primitive invocation is not executed.
It is executed only when the primitive fails, \eg in this case if we try to access a variable that does not exist.
This enables to trigger the debugger on primitive methods; it is possible to modify the code of primitive methods, but beware, this is a quite risky business for the stability of your \sq system.

The following script shows how to display in a \ct!Transcript! window (menu \menu{open ... Transcript}) the contents of the instance variables of a random instance of class \ct!SystemWindow!.
The method \ct!allInstVarNames! returns all the names of the instance variables of a given class.
The method \ct!someInstance! returns the first instance of a class it finds in the memory.

\begin{figure}[ht]\centering
	\includegraphics[width=.75\linewidth]{allInstanceVariables}
	\caption{Displaying all instance variables of a random instance of \ct!SystemWindow!.\label{fig:allInstanceVariables}}
\end{figure}

In the same spirit, it is possible to gather instances that have specific properties.
For instance, to get all instances of class \ct!Browser! whose instance variable \ct!systemOrganizer! is not \ct!nil!, try this expression: \ct!Browser allInstances select: [:c | (c instVarNamed: 'systemOrganizer') notNil]!.

Now let's look at how the method \ct!instanceVariableValues! of \ct!Object! is defined \mthref{instanceVariableValues}. This method returns an array of all values of instance variables defined by this class, excluding the inherited instance variables.
For instance, \ct!(1@2) instanceVariableValues! returns an \ct!OrderedCollection! containing \ct!1! and \ct!2!.
This example shows how useful it is to be able to access instance variables via their position.
The method \ct!instSize! returns the number of instance variables that a class defines.

\begin{method}[instanceVariableValues]{Getting the values of all instance variables}
Object>>>instanceVariableValues
	"Answer a collection whose elements are the values of those instance variables of the receiver which were added by the receiver's class."
	
	| c |
	c := OrderedCollection new.
	self class superclass instSize + 1
		to: self class instSize
		do: [ :i | c add: (self instVarAt: i)].
	^ c
\end{method}

This method iterates over the indexes of instance variables that the class defines, starting just after the last index used by the superclasses.

The development tools in \sq (code browser, debugger, inspector...) all use these reflective features.
Here are a few other methods that might be useful to build development tools:
\begin{itemize}
	\item \ct|isKindOf: aClass| returns true if the receiver is instance of the given class or of one of its superclasses.
	For instance, \ct!1.5 isKindOf: Number! returns \ct!true!, whereas \ct!1.5 isKindOf: Integer! is \ct!false!.
	\item \ct!respondsTo: aSymbol! returns true if the receiver has a method whose selector is \ct!aSymbol!.
	For instance, \ct!1.5 respondsTo: #floor! returns \ct!true!, because class \ct!Number! implements method \ct!floor! that rounds a number.
	Similarly, \ct!Exception respondsTo: #,! because it is possible to group exceptions into a set by sending them the \ct!#,! message.
\end{itemize}

Once again, the reflective features of \sq are only useful to define development tools, and it is risky to use them to develop applications.
Indeed, using the type of an object to drive execution of a method or another often reveals a design problem.
\st offers other introspection possibilities for methods or the execution stack, the process scheduler, the memory manager... but we can't detail everything here; try to read the code and experiment by yourselves.

% section under_the_hood_instances (end)


\section{Browsing the code} % (fold)

In \st, classes are normal objects, and they provide interesting features to navigate live objects.
As we saw previously, you can obtain an instance of a given class by sending it the \ct!#someInstance! message.
You can also gather all the instances with \ct!#allInstances!, or the number of alive instances in memory with \ct!#instanceCount!.
These features are really useful when debugging an application, because you can ask a class to enumerate its methods that have specific properties:
\begin{itemize}
	\item \ct!Point whichSelectorsAccess: 'x'! returns the list of all methods that read the value of instance variable \ct!x! of class \ct!Point! as a set of selectors: \ct!an IdentitySet(#rotateBy:about: #translateBy: #isInsideCircle:with:with: #sideOf: #nearestPointAlongLineFrom:to: #normalized #eightNeighbors #dist: #hash #rotateBy:centerAt: #theta #grid: #fourNeighbors #dotProduct: #scaleFrom:to: #normal #onLineFrom:to:within: #+ #degrees #interpolateTo:at:)!.
	\item \ct!Point whichSelectorsStoreInto: 'x'! returns the selectors of methods that modify the value of instance variable \ct!x!.
	\item \ct!Point whichSelectorsReferTo: #+! returns the selectors of methods that send message \ct!#+!.
	\item \ct!Point crossReference! associates each method with the set of messages it sends.
	\item \ct!Rectangle whichClassIncludesSelector: #inspect! returns the superclass of \ct!Rectangle! that implements the \ct!inspect! method, in this case, \ct!Object!.
	\item \ct!Rectangle unreferencedInstanceVariables! returns the list of instance variables that are not used either in the class or in its subclasses; for \ct!Rectangle! there are no unused instance variables.
\end{itemize}

If you need to navigate the classes in a \sq image, look at the \ct!SystemNavigation! class:
\begin{itemize}
	\item \ct!SystemNavigation default! returns the instance you can use to navigate the system.
	\item \ct!SystemNavigation default allClassesImplementing: #+! returns all classes that implement the message \ct!#+!.
	In this case we get \ct!an Array(Number Fraction Float Integer SmallInteger LargePositiveInteger ScaledDecimal DateAndTime Duration Timespan Player Color Collection WordArray FloatArray KedamaFloatArray String Interval AbstractSound MixedSound Point Voice Complex TraitDescription TraitComposition TraitTransformation TComposingDescription)!.
	As you can see, it is possible to add integers, fractions, strings or intervals.
	\item \ct!SystemNavigation default allSentMessages! returns all messages sent in the \sq image (more than 26000 in our image).
	\item \ct!SystemNavigation default allUnsentMessages! returns all messages (more than 6000...) that have methods implementing them, but are never sent explicitly in the image. These messages are not necessarily useless, because they can be sent implicitely.
	\item \ct!SystemNavigation default allUnimplementedCalls! returns all messages sent but not implemented.
	This is more problematic because they often reveal features that are not implemented in the image.
	\item \ct!SystemNavigation default allCallsOn: #Point! returns all messages sent explicitly to \ct!Point! as a receiver.
\end{itemize}

All these features are integrated in the programming environment of \sq, in particular in the code browsers.
As shown in \secref{cha:env}, you can list all methods that have the same name (implementors) or all methods that send a message (senders).
This is quite handy: for instance, when you want to change the name of a method, to check that you didn't miss one message send that could invoke this method (\figref{implementors}). \damien{is "invoke a method" OK here?}

\begin{figure}[ht]\centering
	\includegraphics[width=.75\linewidth]{implementors}
	\caption{Displaying all classes that implement \ct!\#+!.\label{fig:implementors}}
\end{figure}

To see the internal structure of a class, you can inspect it like any object.
In \figref{CompiledMethod}, the inspector at the back shows the structure of class \ct!Point!.
You can see that the class stores its methods in a dictionary, indexing them by their selector.
The frontmost inspector shows the decompiled bytecode of \ct!Point!'s method \ct!#*!.

\begin{figure}[ht]\centering
	\includegraphics[width=.75\linewidth]{CompiledMethod}
	\caption{Inspecting class \ct!Point! and the bytecode of its \ct!\#*! method.\label{fig:CompiledMethod}}
\end{figure}

\st is really a completely open and transparent system, where all the code is available for reading.
You can study the whole system as you wish, because there is no hidden code.
The primitive part of the system, \ie which is not written in \st, is reduced to its simplest expression.
Among \st dialects, \sq has the peculiarity that even its virtual machine is written in \st, so it is possible to run and debug it inside a \sq image!
Though this actually works, it is quite slow, and for actual use the virtual machine is translated to C and compiled to a native executable.

% section browsing_the_code (end)


\section{Practical uses for introspection} % (fold)

\subsection{Code metrics}

Let's see how we can use \st's introspection features to quickly define code metrics.
A code metric is a measure of program code like depth in the inheritance hierarchy, the number of direct or indeirect subclasses, the number of methods or of instance variables in each class, or the number of locally defined methods or instance variables.
When discovering the code of a new application, you can evaluate a metric to get a rough idea of what the code looks like.
\scrref{metrics} computes a few code metrics for Morphic; these numbers can vary depending on the image you are using.
Results for class \ct!Morph!, which is the superclass of all graphical objects in \sq, reveal that it is a huge class, and that it is at the root of a huge hierarchy. Maybe it needs some refactoring!

\begin{script}[metrics]{A few code metrics}
	inheritanceDepth := Morph allSuperclasses size. "2"
	methods := Morph allSelectors size. "1593"
	instVars := Morph allInstVarNames size. "6"
	addedMethods := Morph selectors size. "1165"
	addedInstVars := Morph instVarNames size. "6"
	directSubclasses := Morph subclasses size. "45"
	totalSubclasses := Morph allSubclasses size. "412"
\end{script}

One of the most interesting metrics in the domain of object-oriented languages is the number of methods that extend methods inherited from the superclass.
This informs us about the relation between the class and its superclasses.

Here is a script that identifies methods in class \ct!Browser! that call the method they redefine, like this:
\begin{code}{}
Browser >>> xx
	"..."
	super xx
\end{code}

\begin{script}[browsersendstosuper]{Methods of \ct!Browser! redefining via \ct!super!}
Browser selectors select: [ :eachSelector |
	| method |
	method := Browser compiledMethodAt: eachSelector.
	method sendsToSuper ]
\end{script}

In this code, we ask to class \ct!Browser! all messages it understands (\ct!selectors!) then select those whose associated method makes a call to the superclass.
This returns a collection of methods, containing for instance \ct!#veryDeepInner!.
You can see that the \st bytecode itself is viewed as an object that we can query for information.
We could even imagine methods that transform this bytecode.


\subsection{Detecting possible errors}

Sending a message to \ct!super! is a bad idea when that message is different from the one the current method implements.
This practice often causes bugs when adding subclasses, so good developers avoids that pattern of code.
It's easy to automatically identify occurences of this pattern: we simply have to find methods that make super-sends with a different selector than their own.

\begin{script}[findSuperSends]{Identifying bad super-sends}
Collection selectors select: [ :eachSelector |
	| method |
	method := Collection compiledMethodAt: eachSelector.
	method sendsToSuper and: [(method messages includes: eachSelector) not]]
\end{script}

Note that this script doesn't find everything we want... it doesn't detect methods containing loops or recursion like the following:
\begin{code}{}
xxx
	super yyy. "send something else than xxx to super"
	self zzz
\end{code}

When executing \scrref{findSuperSends} on class \ct!Collection!, you should obtain method \ct!printNameOn:! which does send \ct!printOn:! to \super:
\begin{method}[printNameOn]{A method that sends a different message to super}
Collection>>>printNameOn: aStream
	super printOn: aStream
\end{method}

In fact this method is used by \ct!printOn:! (\mthref{oldPrintOn}), which could be refactored as in \mthref{newPrintOn}.

\begin{method}[oldPrintOn]{The sender of \lct{\#printNameOn:}}
Collection>>>printOn: aStream
	"Append a sequence of characters that identify the receiver to aStream."
	self printNameOn: aStream.
	self printElementsOn: aStream
\end{method}

\begin{method}[newPrintOn]{New version of \mthref{oldPrintOn}, after inlining the \lct{\#printNameOn:} super-send}
Collection>>>printOn: aStream
	"Append a sequence of characters that identify the receiver to aStream."
	super printOn: aStream.
	self printElementsOn: aStream
\end{method}

Now \ct!#printNameOn:! (\mthref{printNameOn}) is not needed anymore. Well, in this case... is it used in other classes? Let's ask \sq:
\begin{example}[sendersofprintnameon]{Asking for the senders of \lct{\#printNameOn:}}{}
SystemNavigator default allCallsOn: #printNameOn:
--> an OrderedCollection(A MethodReference RunArray >> printOn: aMethodReference Bitmap >> printOn: aMethodReference Text >> printOn: aMethodReference CompiledMethod >> printOn:)
\end{example}

So we should also refactor classes \clsind{RunArray}, \clsind{BitMap}, \clsind{Text}, and \clsind{CompiledMethod} that are sending message \ct!#printNameOn:!.


\subsection{Intelligent breakpoints}

In the 3.8 version of \sq introduced a new kind of breakpoints.
Those are quite useful and are implemented by the \ct!Object>>>haltIf:! method.
This breakpoint only halts program execution if the method it's placed in was called by the one whose name was passed as an argument.
This is very useful to halt a method only during the execution of a test and not every time it is called.
For instance, the expression \ct!self haltIf: #testFoo! will only halt execution of its containing method if this method was called, directly or indirectly, from \ct!testFoo!.
The definition of \ct!haltIf:! is quite simple and fits in five lines; it uses the pseudo-variable \ct!thisContext!, which refers to the execution stack, represented as an object (\ct!thisContext! is one of the six keywords in Smalltalk, with \self, \super, \nil, \ct!true!, and \ct!false!).

\begin{method}[objecthaltif]{Implementation of a conditional breakpoint}
Object>>>haltIf: condition
	"This is the typical message to use for inserting breakpoints during debugging. Param can be a block or expression, halt if true.
	If the condition is a selector, we look up in the callchain. Halt if any method's selector equals selector."
	| cntxt |
	cntxt := thisContext.
	[cntxt sender isNil] whileFalse: [
		cntxt := cntxt sender.
		(cntxt selector = condition) ifTrue: [Halt signal]].
	^self.
\end{method}

Starting from \ct!thisContext!, \ct!haltIf:! goes up through the execution stack, checking if the name of the calling method is the same as the one passed as parameter.
If this is the case, then it raises an exception which, by default, summons the debugger.
This example shows the power of Smalltalk, which allows to define powerful features and tools from within the language itself.

The pseudo-variable \ct!thisContext! is especially used in the \sq debugger.
It contains an instance of the \clsind{MethodContext} class.
This instance contains information about the method that is being executed, the stack pointer (\ct!stackpc!), and the program pointer (\ct!pc!).

In the following example, we halted \sq's evaluation loop by pressing \short{.} and inspecting the current context.

\begin{figure}[ht]\centering
	\includegraphics[width=\linewidth]{MethodContext}
	\caption{Inspecting the execution context of the evaluation loop in \sq.\label{fig:MethodContext}}
\end{figure}

Via \ct!thisContext!, the Seaside web framework also accesses the execution stack, but it modifies it on the fly to easily implement reusable components over HTTP.

% section practical_uses_for_introspection (end)


\section{Chapter summary}

It is natural to use the reflective features in Smalltalk.
Indeed, the whole development environment and its code browsers are built using the introspective interfaces of objects and classes.
We want to stress that it is really interesting to be able to access and modify the objects that represent the program; it makes it unneccessary to devise alternative representations such as abstract syntax trees to build development tools (like Eclipse).
To fix Java~1.0's lack in this respect, Java~1.2 added the ``reflextive'' API ---which is actually only introspective.
Moreover, if the classes and other objects underlying the program execution offer introspective interfaces, then all program representations stay synchronized with the code at all times.
It is thus not needed to keep these representations up-to-date anymore.



%=================================================================
\ifx\wholebook\relax\else\end{document}\fi
%=================================================================
