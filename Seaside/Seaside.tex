% $Author$
% $Date$
% $Revision$
% $Id$
%=================================================================
\ifx\wholebook\relax\else
% --------------------------------------------
% Lulu:
	\documentclass[a4paper,10pt,twoside]{book}
	\usepackage[
		papersize={6in,9in},
		hmargin={.75in,.75in},
		vmargin={.75in,1in},
		ignoreheadfoot
	]{geometry}
	\input{../common.tex}
	\pagestyle{headings}
	\setboolean{lulu}{true}
% --------------------------------------------
% A4:
%	\documentclass[a4paper,11pt,twoside]{book}
%	\input{../common.tex}
%	\usepackage{a4wide}
% --------------------------------------------
    \graphicspath{{figures/} {../figures/}}
	\begin{document}
	% \renewcommand{\nnbb}[2]{} % Disable editorial comments
	\sloppy
\fi
%=================================================================
\chapter{Seaside}
\label{cha:seaside}

\on{I am working on this chapter.}

%=================================================================
\section{XXX}

\begin{verbatim}
Outline
- What is Seaside?
  - Problems with web development
  - advantage of seaside cf IEEE paper
  - history?
  - lots of recent changes
  - Developed by ...
\end{verbatim}

%=================================================================
\section{XXX}
\begin{verbatim}
- Getting Started
  - Downloading and installing
  - Running examples
  - Using the GUI: Halos, debugging etc
\end{verbatim}

%=================================================================
\section{XXX}
\begin{verbatim}
- Overview
  - Components, Tasks, Rendering, Call & Answer ...
\end{verbatim}

%-----------------------------------------------------------------
\subsection{XXX}
\begin{verbatim}
- Rendering
  - Hello World
  - Rendering examples
  - Nested Components
\end{verbatim}
%-----------------------------------------------------------------
\subsection{XXX}
\begin{verbatim}
- Call and Answer
  - Standard Dialogues
  - Tasks
\end{verbatim}
%-----------------------------------------------------------------
\subsection{XXX}
\begin{verbatim}
- Backtracking state
- Transactions
\end{verbatim}

%=================================================================
\section{A complete example}
\begin{verbatim}
- RPN Calculator
\end{verbatim}

%=================================================================
\section{Getting started}

\begin{faq}
How do I install seaside?
\end{faq}
\answer
\begin{itemize}
  \item 
Start the SqueakMap browser.
  \item 
Update map from the net.
  \item 
Install seaside (2.8a)
  \item 
Ignore warnings
  \item 
Say "yes" to "Install KOM server?"
  \item 
Pick (say) admin/seaside for server login and password
Now you probably have an old version of seaside loaded
  \item 
Open Monticello browser
  \item 
Create new repository for http://www.squeaksource.com/Seaside
  \item 
Open and load the latest version of seaside
  \item 
Start the websever ny evaluating:
	WAKom startOn: 8080
  \item 
Point your browser to
	http://localhost:8080/seaside
\end{itemize}

FAQs and other tutorials:
\url{http://www.seaside.st/}

Read also \cite{Duca07a}

Turn server on and off:

\begin{code}
WAKom startOn: 8080
WAKom stop
\end{code}

\begin{faq}
How do you change the login and password?
\end{faq}
\answer

\begin{faq}
How do you turn off the toolbar?
\end{faq}
\answer

\begin{faq}
How do you render components?
\end{faq}
\answer
All components must inherit from WAComponent
Every component must implement \ct{#renderContentOn:}

\begin{faq}
But what about the \ct{#go} method?
\end{faq}
\answer
That's for tasks, not components.

\begin{faq}
What's the difference between a task and a component?
\end{faq}
\answer
It doesn't render -- it just encapsulates a workflow.
(See the class comment.)
Subclasses must not implement \ct{#renderContentOn:}

Root components implement \ct{#canBeRoot} on the class side (returns true)
Applications can be configured through the web interface

Alternatively you can register a component as an application by implementing \ct{#initialize} on the class side:
\begin{code}
initialize
	self registerAsApplication: 'myapp'
\end{code}

Halos ...

\section{Registering state for backtracking in Seaside 2.8}

\ct{#registerObjectForBacktracking:} has been deprecated and actually does not work
anymore in Seaside 2.8


The mechanism in Seaside 2.8 is different, it's similar to how the \ct{#children} mechanism works.

So to define which objects a component is supposed to backtrack, we need to implement a method \ct{#states} which returns an array with these objects.

So for the counter, intead of calling in the initialize

\begin{code}
	self registerObjectForBacktracking: sefl.
\end{code}

we define a method \ct{#states} like this:

\begin{code}
states
	^ Array with: self
\end{code}


For the TicTacToe example of the exercises, this means that we have to change
the class  a little.

\begin{code}
newModel
	self model: STTicTacToeModel empty.
	self session registerObjectForBacktracking: self model board
\end{code}

the line with the \ct{#registerObjectForBacktracking:} has to be deleted. Instead, we add a method \ct{#states:}

\begin{code}
states
	self model ifNil: [^#()].
	^Array with: self model board.
\end{code}


%-----------------------------------------------------------------


%=================================================================
\ifx\wholebook\relax\else 
   \bibliographystyle{jurabib}
   \nobibliography{scg}
   \end{document}
\fi
%=================================================================
